% ==============================================================================
% ==============================       SETS       ==============================
% ==============================================================================
% Requires amsfonts package
\newcommand{\setN}{\mathbb{N}}
\newcommand{\setZ}{\mathbb{Z}}
\newcommand{\setQ}{\mathbb{Q}}
\newcommand{\setR}{\mathbb{R}}
\newcommand{\setC}{\mathbb{C}}


% ==============================================================================
% ===========================       FUNCTIONS       ============================
% ==============================================================================
% Generic function
\newcommand{\fnctn}[2][f]{ #1 \! \left( #2 \right) }

% Expected value
\newcommand{\Ex}[1]{ \operatorname{\mathbb{E}} \! \left[ #1 \right] }
\newcommand{\ExAbs}[1]{ \operatorname{\mathbb{E}} \! \left\vert #1 \right\vert }
\newcommand{\ExCond}[2]{ \Ex{#1 \vert #2} }

% Variance
\newcommand{\Var}[1]{ \operatorname{Var} \! \left[ #1 \right] }

% Covariance & correlation
\newcommand{\Cov}[1]{ \operatorname{Cov} \! \left( #1 \right) }
\newcommand{\Cor}[1]{ \operatorname{Cor} \! \left( #1 \right) }

% Characteristic & moment generating functions
\newcommand{\charfun}[2][X]{ \varphi_{#1} \! \left( #2 \right) }
\newcommand{\charexpfun}[2][X]{ \psi_{#1} \! \left( #2 \right) }
\newcommand{\mgf}[2][X]{ M_{#1} \! \left( #2 \right) }

% Family of sets, functions
\newcommand{\family}[2]{ \left( #1 \right)_{#2} }

% Probability
\newcommand{\prob}[1]{ \operatorname{\probSmb} \! \left( #1 \right) }

% Norm and absolute value
\newcommand{\norm}[1]{ \left\lVert #1 \right\rVert }
\newcommand{\abs}[1]{ \left\vert #1 \right\vert }

% Logarithms
\newcommand{\Log}[2][]{ \operatorname{log}_{#1} \! \left( #2 \right) }
\newcommand{\Ln}[1]{ \operatorname{ln} \! \left( #1 \right) }

% Indicator function
\newcommand{\indicator}[2][x]{ \mathbbm{1}_{#2} \! \left( #1 \right) }

% Exponent function
\newcommand{\Exp}[1]{ \operatorname{exp} \! \left( #1 \right) }

% Real and imaginary part
\renewcommand{\Re}[1]{ \operatorname{Re} \! \left( #1 \right) }
\renewcommand{\Im}[1]{ \operatorname{Im} \! \left( #1 \right) }

% Interior and closure
\newcommand{\interior}[1]{ \operatorname{int} \! \left( #1 \right) }
\newcommand{\closure}[1]{ \overline{#1} }

% Gamma functions
\newcommand{\fGamma}[1]{ \operatorname{\Gamma} \! \left( #1 \right) }
\newcommand{\fUGamma}[2]{ \operatorname{\Gamma} \! \left( #1,#2 \right) }
\newcommand{\fLGamma}[2]{ \operatorname{\gamma} \! \left( #1,#2 \right) }


% ==============================================================================
% ======================       STRUCTURES AND OTHER       ======================
% ==============================================================================
% Probability space
\newcommand{\measureSpace}{ \left( \Omega, \sigmaAlgebra \right) }
\newcommand{\probSpace}{ \left( \Omega, \sigmaAlgebra, \probSmb \right) }
\newcommand{\filtProbSpace}{ \left( \Omega, \sigmaAlgebra, \mathbb{F}, \probSmb \right) }

% Big for all
\newcommand{\bigforall}{ \mathop{\mbox{\Large $\mathsurround=0pt\forall$}} }
\newcommand{\bigexists}{ \mathop{\mbox{\Large $\mathsurround=0pt\exists$}} }

% Integrals
\newcommand{\evalAt}[3]{ \! \left. #1 \right\vert_{#2}^{#3} }
\newcommandx{\integral}[2][1={}, 2={}]{ \int\limits_{#1}^{#2} }

% Create a set
\newcommand{\set}[1]{ \left\lbrace #1 \right\rbrace}
\newcommand{\setCond}[2]{ \left\lbrace #1 : #2 \right\rbrace}

% Radon-Nikodym derivative
\newcommand{\RNDerivative}[2]{ \dfrac{d #1}{d #2}}

% Borel sets
\newcommand{\borelSet}[1]{ \borelSmb \! \left( #1 \right) \! }

% Stochastic convergence
\newcommand{\stochConv}[1][\probSmb]{ \xrightarrow{#1} }

% Function definition
\newcommand{\fundef}[3]{ #1 \colon #2 \to #3 }


% ==============================================================================
% =========================       DISTRIBUTIONS       ==========================
% ==============================================================================
% Probability distributions
\newcommandx{\distNorm}[2][1=\mu, 2=\sigma^2]{ \mathcal{N} \! \left( #1, #2 \right) }
\newcommandx{\distPois}[1][1=\lambda]{ \operatorname{Pois} \! \left( #1 \right) }
\newcommandx{\distGamma}[2][1=\alpha, 2=\beta]{ \Gamma \! \left( #1, #2 \right) }
\newcommandx{\distCompPois}[2][1=\lambda, 2=F]{ \operatorname{CP} \! \left( #1, #2 \right) }


% ==============================================================================
% =========================       MISCELLANEOUS       ==========================
% ==============================================================================
% Handles input tables from R xtable package (different encoding)
% \newcommand{\xtable}[1]{\inputencoding{latin1}\input{#1}}

% To mark text as a comment
\newcommand{\COMMENTBOX}[1]{ \todo[inline, backgroundcolor=blue!20]{#1} }

% To mark text as important/TODO
\newcommand{\TODOBOX}[1]{ \todo[inline]{#1} }


% ==============================================================================
% ============================       COLOURS       =============================
% ==============================================================================
\definecolor{skyblue}{HTML}{6CA6CD}
\definecolor{salmon}{HTML}{F05D62}

% For Beamer presentation
\definecolor{tugreen}{rgb}{0.5686275,0.6745098,0.4196078}
\definecolor{tulightgreen}{rgb}{ 0.7098039,0.7921569,0.5098039}
\definecolor{tugrey}{rgb}{0.6117647,0.6156863,0.6235294}
\definecolor{tured}{rgb}{0.8980392,0.2039216,0.09411765}
\definecolor{tublue}{rgb}{0,0.3,1}
\definecolor{tulightgrey}{rgb}{0.99,0.99,0.99}
\definecolor{ecs100}{RGB}{93,147,189}

\newcommand{\Blue}{\color{tublue}}
\newcommand{\Red}{\color{tured}}
\newcommand{\tcrb}{\color{tublue}}
\newcommand{\tcrg}{\color{tugreen}}
\newcommand{\tcrr}{\color{tured}}
\newcommand{\tcro}{\color{orange}}
\newcommand{\tcrm}{\color{magenta}}
\newcommand{\tcrlg}{\color{tulightgreen}}
\newcommand{\tcrgr}{\color{tugrey}}


% ==============================================================================
% ===================       FIXED STRINGS AND SYMBOLS       ====================
% ==============================================================================
% Symbols
\newcommand{\probSmb}{ \mathbb{P} }
% For sigma-algebras
\newcommand{\sigmaAlgebra}{ \mathcal{F} }
% Borel
\newcommand{\borelSmb}{ \mathcal{B} }


% ==============================================================================
% ==============================       END       ===============================
% ==============================================================================
