% ==============================================================================
% ============================       THEOREMS       ============================
% ==============================================================================

% Notes ------------------------------------------------------------------------
% Every \counterwithin call must be made after tcolorbox definitions
%
% Legend for naming:
% d*: default environment,
% f*: framed environment.


% Settings ---------------------------------------------------------------------
% Counters
\newcounter{definitioncounter}
\newcounter{examplecounter}
\newcounter{theoremcounter}
\newcounter{lemmacounter}
\newcounter{remarkcounter}

% Use library from tcolorbox package
\tcbuselibrary{theorems}


% Bolded title -----------------------------------------------------------------
\theoremstyle{definition}

% Definition
\newtheorem{ddefinition}[definitioncounter]{\definitionname}
\newtcbtheorem[number within=section,
			   use counter=definitioncounter]
	{fdefinition} % Environment name
	{\definitionname} % Displayed name
	{colback=definitioncolor!7,
	 colframe=definitioncolor,
	 fonttitle=\bfseries}
	{def} % Label prefix

% Example
\newtheorem{dexample}[examplecounter]{\examplename}
\newtcbtheorem[number within=section,
			   use counter=examplecounter]
	{fexample} % Environment name
	{\examplename} % Displayed name
	{colback=examplecolor!7,
	 colframe=examplecolor,
	 fonttitle=\bfseries}
	{ex} % Label prefix


% Bolded title, italic text ----------------------------------------------------
\theoremstyle{plain}

% Theorem
\newtheorem{dtheorem}[theoremcounter]{\theoremname}
\newtcbtheorem[number within=section,
			   use counter=theoremcounter]
	{ftheorem} % Environment name
	{\theoremname} % Displayed name
	{colback=theoremcolor!7,
	 colframe=theoremcolor,
	 fonttitle=\bfseries,
	 fontupper=\slshape}
	{th} % Label prefix

% Corollary
\newtheorem{dcorollary}[theoremcounter]{\corollaryname}
\newtcbtheorem[number within=section,
			   use counter=theoremcounter]
	{fcorollary} % Environment name
	{\corollaryname} % Displayed name
	{colback=theoremcolor!7,
	 colframe=theoremcolor!50!lemmacolor,
	 fonttitle=\bfseries,
	 fontupper=\slshape}
	{cor} % Label prefix

% Lemma
\newtheorem{dlemma}[lemmacounter]{\lemmaname}
\newtcbtheorem[number within=section,
	           use counter=lemmacounter]
	{flemma} % Environment name
	{\lemmaname} % Displayed name
	{colback=lemmacolor!7,
	 colframe=lemmacolor,
	 fonttitle=\bfseries,
	 fontupper=\slshape}
	{lem} % Label prefix


% Italic title -----------------------------------------------------------------
\theoremstyle{remark}

% Remark
\newtheorem{dremark}[remarkcounter]{\remarkname}
\newtcbtheorem[number within=section,
               use counter=remarkcounter]
    {fremark} % Environment name
    {\remarkname} % Displayed name
    {colback=remarkcolor!7,
     colframe=remarkcolor,
     fonttitle=\slshape}
    {rem} % Label prefix


% Reset counters to section ----------------------------------------------------
\counterwithin{definitioncounter}{section}
\counterwithin{examplecounter}{section}
\counterwithin{theoremcounter}{section}
\counterwithin{lemmacounter}{section}
\counterwithin{remarkcounter}{section}
